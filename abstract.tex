\chapter*{Abstract}

The number of artificial objects in the near-Earth space realm is on a consistent rise each year,
posing a serious threat to the safety of operational space assets. Effective Space Situational Awaren-
ess (SSA) practices are vital to continuously update our understanding of the space environment.
This knowledge is essential for implementing appropriate measures to mitigate risks in densely
populated orbits and to enhance sustainability in the near-space domain. In order to mitigate these
risks, continuous monitoring of the space region and building space debris object catalogs is
necessary. To ensure the high accuracy of catalogued objects, the follow-up observations consisting
orbit improvement information are needed. Given the constraints of observation resources, espe-
cially for space debris within the geostationary orbit, where observational windows during the night
are restricted, and considering telescope capabilities, effective sensor tasking of the available sensor network
holds significant importance.\\

This thesis delves into the examination of global sensor networks for space debris observations. It integrates various weighting schemes into the cost function of 
the Greedy algorithm, a tool employed for surveying in the SNS software. The research introduces the application of a genetic algorithm to optimize follow-up observation 
scheduling within a telescope network, with a focal point on catalog maintenance. The primary optimization parameter is the information content associated with the orbit. 
Emphasis is placed on the significance of utilizing orbit covariance and quantifying information gain for effective optimization.
The study concludes with simulations conducted in the Sensor Network Simulator (SNS), a tool suite developed to meticulously study and optimize the performance of existing sensors and their configurations. The effectiveness of the algorithm is analyzed within this framework.

\pagebreak

\chapter*{Kurzfassung}

Die Zahl der künstlichen Objekte im erdnahen Weltraum nimmt jedes Jahr zu,
Sie stellen eine ernsthafte Bedrohung für die Sicherheit der operativen Weltraumressourcen dar. Effektive SSA ist unerlässlich, um unser Verständnis der Weltraumumgebung ständig zu aktualisieren.
Dieses Wissen ist entscheidend für die Umsetzung geeigneter Maßnahmen zur Risikominderung in dicht besiedelten
dicht besiedelten Orbits und zur Verbesserung der Nachhaltigkeit im Nahbereich des Weltraums. Um diese Risiken zu mindern
Risiken zu mindern, ist eine kontinuierliche Überwachung des Weltraums und die Erstellung von Katalogen von Weltraummüllobjekten
notwendig. Um die hohe Genauigkeit der katalogisierten Objekte zu gewährleisten, sind Folgebeobachtungen, die
Informationen zur Verbesserung der Umlaufbahn benötigt werden. Angesichts der begrenzten Beobachtungsressourcen, insbesondere
insbesondere für Weltraummüll in der geostationären Umlaufbahn, wo die Beobachtungsfenster während der Nacht
und unter Berücksichtigung der Teleskopkapazitäten ist eine effektive Nutzung des verfügbaren Sensornetzes
von großer Bedeutung.\\

Diese Arbeit befasst sich mit der Untersuchung globaler Sensornetzwerke zur Beobachtung von Weltraummüll. Sie integriert verschiedene Gewichtungsschemata in die Kostenfunktion des 
des Greedy-Algorithmus, der in der SNS-Software zur Vermessung eingesetzt wird. Die Forschung stellt die Anwendung eines genetischen Algorithmus zur Optimierung der Nachbeobachtung 
Beobachtungsplanung innerhalb eines Teleskopnetzwerks, wobei der Schwerpunkt auf der Katalogpflege liegt. Der primäre Optimierungsparameter ist der mit der Umlaufbahn verbundene Informationsgehalt. 
Der Schwerpunkt liegt auf der Bedeutung der Nutzung der Orbit-Kovarianz und der Quantifizierung des Informationsgewinns für eine effektive Optimierung.
Die Studie schließt mit Simulationen, die mit dem Sensor Network Simulator (SNS) durchgeführt wurden, einem Toolpaket, das entwickelt wurde, um die Leistung bestehender Sensoren und ihrer Konfigurationen genau zu untersuchen und zu optimieren. Die Effektivität des Algorithmus wird in diesem Rahmen analysiert.