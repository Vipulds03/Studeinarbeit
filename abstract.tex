\chapter*{Abstract}

The number of artificial objects in the near-Earth space realm is on a consistent rise each year,
posing a serious threat to the safety of operational space assets. Effective Space Situational Awareness (SSA) practices are vital to continuously update our understanding of the space environment.
This knowledge is essential for implementing appropriate measures to mitigate risks in densely
populated orbits and to enhance sustainability in the near-space domain. In order to mitigate these
risks, continuous monitoring of the space region and building space debris object catalogs is
necessary. To ensure the high accuracy of catalogued objects, the follow-up observations consisting
orbit improvement information are needed. Given the constraints of observation resources, especially for space debris within the geostationary orbit, where observational windows during the night
are restricted, and considering telescope capabilities, effective sensor tasking of the available sensor network
holds significant importance.\\

This thesis delves into the examination of global sensor networks for space debris observations. It integrates various weighting schemes into the cost function of 
the Greedy algorithm, a tool employed for surveying in the SNS software. The research also briefly introduces the application of a genetic algorithm to optimize follow-up observation 
scheduling within a telescope network, with a focal point on catalog maintenance. The primary optimization parameter is the information content associated with the orbit. 
Emphasis is placed on the significance of utilizing orbit covariance and quantifying information gain for effective optimization. The implementation of the genetic algorithm is not a component of this study.
The study concludes with simulations conducted in the Sensor Network Simulator (SNS), a tool suite developed to meticulously study and optimize the performance of existing sensors and their configurations. The effectiveness of the algorithm is analyzed within this framework.

\pagebreak

\chapter*{Kurzfassung}

Die Anzahl künstlicher Objekte im Bereich des erdnahen Weltraums nimmt jedes Jahr kontinuierlich zu und stellt eine ernsthafte Bedrohung für die Sicherheit operativer Weltraumressourcen dar. Effektive Praktiken zum Space Situational Awareness, (SSA) sind entscheidend, um unser Verständnis der Weltraumumgebung kontinuierlich zu aktualisieren. Dieses Wissen ist unerlässlich, um geeignete Maßnahmen zur Risikominderung in dicht besiedelten Umlaufbahnen umzusetzen und die Nachhaltigkeit im Bereich des erdnahen Weltraums zu fördern. Um diese Risiken zu minimieren, ist eine kontinuierliche Überwachung der Weltraumregion und der Aufbau von Katalogen für Weltraumtrümmerobjekte notwendig. Um die hohe Genauigkeit der katalogisierten Objekte sicherzustellen, sind Nachbeobachtungen mit Informationen zur Verbesserung der Umlaufbahn erforderlich. Angesichts der Einschränkungen der Beobachtungsressourcen, insbesondere für Weltraumtrümmer in der geostationären Umlaufbahn, wo die Beobachtungsfenster während der Nacht eingeschränkt sind, und unter Berücksichtigung der Teleskopfähigkeiten, hat eine effektive Sensorausrichtung des verfügbaren Sensornetzwerks eine erhebliche Bedeutung.\\

Diese Arbeit widmet sich der Untersuchung globaler Sensornetzwerke für die Beobachtung von Weltraummüll. Sie integriert verschiedene Gewichtungsschemata in die Kostenfunktion des Greedy-Algorithmus, eines Instruments, das für Umfragen in der SNS-Software verwendet wird. Die Forschung führt auch kurz die Anwendung eines genetischen Algorithmus zur Optimierung der Terminplanung für Nachbeobachtungen in einem Teleskopnetzwerk ein, wobei der Schwerpunkt auf der Katalogwartung liegt. Der primäre Optimierungsparameter ist der Informationsgehalt, der mit der Umlaufbahn verbunden ist. Betont wird die Bedeutung der Verwendung von Umlaufbahndispersion und der Quantifizierung des Informationsgewinns für eine effektive Optimierung. Die Implementierung des genetischen Algorithmus ist kein Bestandteil dieser Studie. Die Studie schließt mit Simulationen ab, die im Sensor Network Simulator (SNS), einer Tool-Suite zur gründlichen Untersuchung und Optimierung der Leistung bestehender Sensoren und ihrer Konfigurationen, durchgeführt wurden. Die Wirksamkeit des Algorithmus wird innerhalb dieses Rahmens analysiert.\\