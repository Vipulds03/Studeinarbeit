\chapter{Theoretische Grundlagen}
\label{sec:grundlagen}

\section{Literaturrecherche}

Zu jeder guten Arbeit gehört eine umfangreiche Literaturrecherche. Während in den ersten Gesprächen mit dem Betreuer bereits einige Tipps zu empfohlener 
Literatur erhalten werden, ist es in den meisten Fällen notwendig, selbst nach Literatur zu suchen. Dazu bietet sich zunächst die UB an, mit der man als 
Student bereits vertraut sein sollte. Darüber hinaus ist folgendes Vorgehen zu empfehlen:
\begin{itemize}
 \item Suche nach Literatur, auf die in den Referenzen bereits erhaltener Literatur verwiesen wird.
 \item Suche bei \textit{Science Direct}, wobei zahlreiche Paper aus dem Uni-Netzwerk kostenlos heruntergeladen werden können: 
\url{http://www.sciencedirect.com/}
 \item Suche in digitaler Bibliothek des ILR
\end{itemize}
Für den letzteren Punkt wird ein Account am ILR benötigt, den man beim Betreuer beantragen kann. Die Nutzung der digitalen Bibliothek funktioniert 
folgendermaßen:
\begin{enumerate}
  \item Kopiere die Datei \tb{JabRef-2.10.jar} z.B. auf deinen Desktop (oder unter Linux \tb{JabRef} nach \ts{~/bin/}.). Hier ggf. den Betreuer nach dem 
aktuellen Pfad der Datei fragen.
  \item Durch anklicken kann JabRef jetzt gestartet werden (oder unter Linux durch den Befehl \tb{JabRef} aus der Konsole)
  \item Oben links befindet sich der Button zum Öffnen: \ts{Öffnen einer BibTex-Database}
  \item Öffne die Datei \ts{/home/export/biblio/e-biblio.bib} (unter Windows ggf. das Laufwerk \ts{biblio an ILR-Server (Jupiter)} einbinden)
  \item Die elektronische Bibliothek ist geöffnet (diese wird beim nächsten Starten von JabRef automatisch wieder geladen)
  \item Zugriff auf die Literatur:
  \begin{itemize}
   \item Elektronische Literatur lässt sich direkt aus JabRef öffnen
   \item Literatur, die nur als Buch oder Kopiervorlage existiert, kann über das Feld \ts{Note} gefunden werden, in dem die entsprechende Referenznummer 
abgelegt ist. Hier am besten den Betreuer fragen, wo im Institut sich das Dokument dann befindet (Archiv).
  \end{itemize}
\end{enumerate}

\section{Bewertung der Arbeit}

Alle Arbeiten am ILR werden nach einem vorgegebenen Schema bewertet, so dass vorkommen kann, dass Studenten, die eine perfekte Software als Lösung
des Problems liefern, oder etwa eine perfekte Konstruktionslösung bieten, dennoch Abzüge in der Note erhalten können,
wenn etwa die Dokumentation Schwächen zeigt (äußere Form, \ts{roter Faden} in der Arbeit, Unvollständigkeit, etc.).

Die Beurteilungskriterien sind in der folgenden Tabelle aufgelistet:
\begin{table}[h!]
 \centering
 \begin{tabular}{lr}
  Inhalt (Einführung, Theorie, Lösungsweg, Ergebnisse, Diskussion) & 50\% \\
  Äußere Form der Arbeit & 15\% \\
  Arbeitsweise (Eigene Ideen, Selbständigkeit, Methodik) & 30\% \\
  Zeiteinteilung & 5\% \\
 \end{tabular}

\end{table}

\section{Formelzeichen}

Ein Symbol- und Abk\"urzungsverzeichnis sind, neben den bereits automatisch 
durch diese Vorlage erstellten Tabellen- und Abbildungsverzeichnissen, zu erstellen.
Dabei wird empfohlen, von dem Paket \tb{glossaries} Gebrauch zu machen, welches 
ebenfalls in dieser Vorlage eingebunden ist. Hier ein paar Beispiele:
\begin{itemize}
 \item Ein Symbol, welches im Symbolverzeichnis auftauchen soll: \gls{symb:J2}
 \item Eine Abk\"urzung wird beim ersten Mal ausgeschrieben: \gls{ILR} und ab dem zweiten Mal nicht mehr: \gls{ILR}
\end{itemize}
In Gleichungen verwendet man die Formelzeichen analog, wie an \glg{eq:beispiel} zu sehen ist:
\begin{equation}
 \gls{symb:E} = \gls{symb:m}\cdot \gls{symb:c}^2 \label{eq:beispiel}
\end{equation}

Zunächst werden die als Formelzeichen benutzten lateinischen bzw. deutschen Buchstaben in alphabetischer Reihenfolge geordnet, wobei
jeweils die großen Buchstaben den kleinen voranzustellen sind. Dann in gleicher
Reihenfolge die dem griechischen Alphabet entnommenen Symbole. Am Schluss
sind die häufig benutzten Indizes in alphabetischer Reihenfolge anzugeben. Beinhaltet
die Arbeit die Erstellung eines Programms oder von Teilen hierzu, so sollen die im
Programm verwendeten Bezeichnungen den Formelzeichen zugeordnet werden. Es
sind ausschließlich SI-Einheiten zu verwenden.

Die Sortierung kann ebenfalls automatisch vorgenommen werden, indem im Symbolverzeichnis (Datei: \ts{symbolverzeichnis.tex}) jedem Eintrag der entsprechende 
\tb{sort}-Schlüssel gegeben wird, z.B.:
\begin{itemize}
 \item Das Symbol \tb{c} soll im Symbolverzeichnis auftauchen. Es handelt sich um einen lateinischen Buchstaben, 
       der Schlüssel zum Sortieren wird mit \ts{sort=xc} vergeben (Präfix \ts{x}, der gleich ersichtlich wird - s.u.).
 \item Das Symbol \tb{$\alpha$} ist ein griechischer Buchstabe und soll erst nach den lateinischen Buchstaben kommen. 
       Entsprechend stellt man den Schlüssel auf \ts{sort=yalpha} (Präfix \ts{y}, sodass griechische Buchstaben immer
       nach den lateinischen Buchstaben erscheinen werden.
\end{itemize}

Existieren benötigte Herleitungen bereits in der Literatur, so ist auf diese zu verweisen. Nur zum unmittelbaren Verständnis notwendige Gleichungen sollten
dargestellt werden. Es ist darauf zu achten, dass sonstige theoretische Herleitungen schlüssig sind (kontrollieren, ob jede Größe eindeutig definiert ist
und keine Gedankensprünge vorhanden sind). Herleitungen soweit wie möglich mit allgemeinen Bezeichnungen durchführen, Zahlenwerte erst im Ergebnisteil (ab 
\kap{sec:hauptteil}) einsetzen.